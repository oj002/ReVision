\documentclass[10pt,a4paper]{article}
\usepackage[utf8]{inputenc}
\usepackage{amsmath}
\usepackage{amsfonts}
\usepackage{amssymb}
\usepackage{listings}
\author{Olaf Bernstein}
\title{ReVision}

\begin{document}
\maketitle
\tableofcontents

\section{Wishlist}


\subsection{Keywords}
\begin{tabular}{||c|l||}\hline
	keyword 	& description	\\ \hline \hline
 
  
	use 	 	& You will be able to list specific variables you want to use and \\
  				& the scope of the next code block gets reduced to this variables. \\
  				& Should be able to return values, the return type will get deduced at compile time. \\
&\begin{lstlisting} 
a = use x, y {
	...
	// return from the use block
  	return 3;
};
\end{lstlisting} \\


	namespace & C++ Style namespace keyword. \\
	\\\hline
\end{tabular}


\subsection{Type System}
The type system will be strongly typed with something similar to the C++ keyword \textit{auto} and will use static type checking.


\subsubsection{Primitive Types}
\begin{tabular}{||c|c||}\hline
	signed & usigned	\\ \hline \hline
	--- 	 & char		\\
	int8   & uint8		\\
	int16  & uint16		\\
	int32  & uint32		\\
	int64  & uint64		\\
	float  & ---		\\
	double & ---		\\
\hline
\end{tabular}


\subsubsection{Composite Types}
\begin{tabular}{||c|l||}\hline
	type 		& description					\\ \hline \hline
	union 	& Like the C/C++ union type. 		\\
	struct 	& Like the C struct.		   		\\
	class 	& C++ class type with some changes:	\\
			&  Private members variables are accessible but can't be changed. \\
			&  Functions must be separated from the member variables. \\
	\hline
\end{tabular}
  

\subsubsection{Type Qualifier}
These work for all types above:\\
\begin{tabular}{||c|l||}\hline
	qualifier & description							\\ \hline \hline
	const 	& Like the C/C++ const type qualifier. 	\\
	\hline
\end{tabular}


\subsubsection{Other}
\begin{tabular}{||c|l||}\hline
	keyword/operator 	& description				\\ \hline \hline
	'*' type 			& Like the C/C++ pointer. 	\\
	'\&' type 			& Like the C/C++ reference. \\
	\hline
\end{tabular}


\subsection{Functions}
Can be declared in the global namespace but also inside a function which will only be accessible in the local space. Forward declarations similar to C/C++ will be supported and operator overloading.


\subsection{Operator}


\end{document}
