\documentclass[10pt,a4paper]{article}
\usepackage[utf8]{inputenc}
\usepackage{amsmath}
\usepackage{amsfonts}
\usepackage{amssymb}
\usepackage{listings}
\author{Olaf Bernstein}
\title{ReVision Goals}

\begin{document}

\maketitle
The Goal of this project is the complete revision of my favourite concepts of programming languages. It will be primarily oriented around C++ concepts because its currently one of my favourite and most used programming languages.
\tableofcontents




\section{Wy Not C++?}
So why not C++, well C++ might support a grate amount of features, heck you can even make thinks like an NES emulator at compile time, but has gotten pretty messy lately. For Example, there are still trying to keep the C backwards compatibility, but have extremely many new features. Some serve the same purpose of the C equivalent, but are only there to support the new C++ features.




\section{Control Flow}


\subsection{\textit{while}}
1. If condition is true goto 3. \\
2. Execute code block. \\
3. goto 1. \\
4. Continue execution. \\
Syntax:
\begin{lstlisting}
while(condition)
{
	// ...
}
\end{lstlisting}


\subsection{\textit{goto}}


\subsection{\textit{for}}


\subsection{\textit{switch}}


\subsection{\textit{if}}


\subsection{\textit{do}}
1. Execute code block. \\
2. If condition is true goto 1. \\
4. Continue execution. \\
Syntax:
\begin{lstlisting}
do
{
	// ...
} while(condition);
\end{lstlisting}


\subsection{\textit{use}}
The scope of the code block gets reduced to the variables in the capture list.
\begin{lstlisting}
use x, y // Capture list
{
	// ...
};
\end{lstlisting}
Optional you can return a variable just like in a function.
\begin{lstlisting}
z = use x, y // Capture list
{
	// ...
	return val;
};
\end{lstlisting}


\subsection{\textit{asm}}





\section{Composite Types}


\subsection{\textit{class}}


\subsection{\textit{enum}}


\subsection{\textit{uniom}}


\subsection{\textit{string}}





\section{Type Modifier}


\subsection{\textit{const}}





\section{Primitive Types}


\subsection{Integer}


\subsection{Floating Point}

\subsubsection{\textit{float}}

\subsubsection{\textit{double}}


\subsection{Other}

\subsubsection{\textit{char}}

\subsubsection{\textit{auto}}





\section{Other}


\subsection{\textit{new}}


\subsection{\textit{delete}}


\subsection{\textit{namespace}}




\end{document}
